% Options for packages loaded elsewhere
% Options for packages loaded elsewhere
\PassOptionsToPackage{unicode}{hyperref}
\PassOptionsToPackage{hyphens}{url}
\PassOptionsToPackage{dvipsnames,svgnames,x11names}{xcolor}
%
\documentclass[
  10pt,
  a4paper,
  DIV=11,
  numbers=noendperiod]{scrartcl}
\usepackage{xcolor}
\usepackage[left=15mm,right=15mm,top=20mm,bottom=20mm]{geometry}
\usepackage{amsmath,amssymb}
\setcounter{secnumdepth}{-\maxdimen} % remove section numbering
\usepackage{iftex}
\ifPDFTeX
  \usepackage[T1]{fontenc}
  \usepackage[utf8]{inputenc}
  \usepackage{textcomp} % provide euro and other symbols
\else % if luatex or xetex
  \usepackage{unicode-math} % this also loads fontspec
  \defaultfontfeatures{Scale=MatchLowercase}
  \defaultfontfeatures[\rmfamily]{Ligatures=TeX,Scale=1}
\fi
\usepackage{lmodern}
\ifPDFTeX\else
  % xetex/luatex font selection
  \setmainfont[]{Candara}
\fi
% Use upquote if available, for straight quotes in verbatim environments
\IfFileExists{upquote.sty}{\usepackage{upquote}}{}
\IfFileExists{microtype.sty}{% use microtype if available
  \usepackage[]{microtype}
  \UseMicrotypeSet[protrusion]{basicmath} % disable protrusion for tt fonts
}{}
\makeatletter
\@ifundefined{KOMAClassName}{% if non-KOMA class
  \IfFileExists{parskip.sty}{%
    \usepackage{parskip}
  }{% else
    \setlength{\parindent}{0pt}
    \setlength{\parskip}{6pt plus 2pt minus 1pt}}
}{% if KOMA class
  \KOMAoptions{parskip=half}}
\makeatother
% Make \paragraph and \subparagraph free-standing
\makeatletter
\ifx\paragraph\undefined\else
  \let\oldparagraph\paragraph
  \renewcommand{\paragraph}{
    \@ifstar
      \xxxParagraphStar
      \xxxParagraphNoStar
  }
  \newcommand{\xxxParagraphStar}[1]{\oldparagraph*{#1}\mbox{}}
  \newcommand{\xxxParagraphNoStar}[1]{\oldparagraph{#1}\mbox{}}
\fi
\ifx\subparagraph\undefined\else
  \let\oldsubparagraph\subparagraph
  \renewcommand{\subparagraph}{
    \@ifstar
      \xxxSubParagraphStar
      \xxxSubParagraphNoStar
  }
  \newcommand{\xxxSubParagraphStar}[1]{\oldsubparagraph*{#1}\mbox{}}
  \newcommand{\xxxSubParagraphNoStar}[1]{\oldsubparagraph{#1}\mbox{}}
\fi
\makeatother


\usepackage{longtable,booktabs,array}
\usepackage{calc} % for calculating minipage widths
% Correct order of tables after \paragraph or \subparagraph
\usepackage{etoolbox}
\makeatletter
\patchcmd\longtable{\par}{\if@noskipsec\mbox{}\fi\par}{}{}
\makeatother
% Allow footnotes in longtable head/foot
\IfFileExists{footnotehyper.sty}{\usepackage{footnotehyper}}{\usepackage{footnote}}
\makesavenoteenv{longtable}
\usepackage{graphicx}
\makeatletter
\newsavebox\pandoc@box
\newcommand*\pandocbounded[1]{% scales image to fit in text height/width
  \sbox\pandoc@box{#1}%
  \Gscale@div\@tempa{\textheight}{\dimexpr\ht\pandoc@box+\dp\pandoc@box\relax}%
  \Gscale@div\@tempb{\linewidth}{\wd\pandoc@box}%
  \ifdim\@tempb\p@<\@tempa\p@\let\@tempa\@tempb\fi% select the smaller of both
  \ifdim\@tempa\p@<\p@\scalebox{\@tempa}{\usebox\pandoc@box}%
  \else\usebox{\pandoc@box}%
  \fi%
}
% Set default figure placement to htbp
\def\fps@figure{htbp}
\makeatother





\setlength{\emergencystretch}{3em} % prevent overfull lines

\providecommand{\tightlist}{%
  \setlength{\itemsep}{0pt}\setlength{\parskip}{0pt}}



 


\KOMAoption{captions}{tableheading}
\makeatletter
\@ifpackageloaded{caption}{}{\usepackage{caption}}
\AtBeginDocument{%
\ifdefined\contentsname
  \renewcommand*\contentsname{Table of contents}
\else
  \newcommand\contentsname{Table of contents}
\fi
\ifdefined\listfigurename
  \renewcommand*\listfigurename{List of Figures}
\else
  \newcommand\listfigurename{List of Figures}
\fi
\ifdefined\listtablename
  \renewcommand*\listtablename{List of Tables}
\else
  \newcommand\listtablename{List of Tables}
\fi
\ifdefined\figurename
  \renewcommand*\figurename{Figure}
\else
  \newcommand\figurename{Figure}
\fi
\ifdefined\tablename
  \renewcommand*\tablename{Table}
\else
  \newcommand\tablename{Table}
\fi
}
\@ifpackageloaded{float}{}{\usepackage{float}}
\floatstyle{ruled}
\@ifundefined{c@chapter}{\newfloat{codelisting}{h}{lop}}{\newfloat{codelisting}{h}{lop}[chapter]}
\floatname{codelisting}{Listing}
\newcommand*\listoflistings{\listof{codelisting}{List of Listings}}
\makeatother
\makeatletter
\makeatother
\makeatletter
\@ifpackageloaded{caption}{}{\usepackage{caption}}
\@ifpackageloaded{subcaption}{}{\usepackage{subcaption}}
\makeatother
\usepackage{bookmark}
\IfFileExists{xurl.sty}{\usepackage{xurl}}{} % add URL line breaks if available
\urlstyle{same}
\hypersetup{
  pdftitle={André Luiz Barbosa Nunes da Cunha, Ph.D.},
  pdfauthor={Assistant Professor of Civil Engineering},
  colorlinks=true,
  linkcolor={blue},
  filecolor={Maroon},
  citecolor={Blue},
  urlcolor={blue},
  pdfcreator={LaTeX via pandoc}}


\title{André Luiz Barbosa Nunes da Cunha, Ph.D.}
\author{Assistant Professor of Civil Engineering}
\date{}
\begin{document}
\maketitle


\section{\texorpdfstring{PERSONAL DETAILS
\hrulefill}{PERSONAL DETAILS }}\label{personal-details}

\begin{description}
\tightlist
\item[E-mail]
\href{mailto:alcunha@usp.br}{\nolinkurl{alcunha@usp.br}}
\item[Phone]
+55 16 98119 2339
\item[Office Address]
Departament of Transport Engineering,

University of São Paulo (\href{https://www5.usp.br/en}{USP}), São Carlos
School of Engineering (\href{https://eesc.usp.br/en/}{EESC}),

São Carlos, São Paulo, Brazil
\item[Key links]
\href{https://scholar.google.com.br/citations?user=HI0CQJMAAAAJ&hl=en}{Google
Scholar} \footnote{\url{https://scholar.google.com.br/citations?user=HI0CQJMAAAAJ&hl=en}}
\textbar{} \href{https://orcid.org/0000-0002-0520-0621}{ORCID}
\footnote{\url{https://orcid.org/0000-0002-0520-0621}} \textbar{}
\href{https://www.webofscience.com/wos/author/record/U-4375-2019}{WoS}
\footnote{\url{https://www.webofscience.com/wos/author/record/U-4375-2019}}
\textbar{} \href{https://www.linkedin.com/in/prof-alcunha/}{LinkedIn}
\footnote{\url{https://www.linkedin.com/in/prof-alcunha/}} \textbar{}
\href{https://linktr.ee/prof_alcunha}{LinkTree} \footnote{\url{https://linktr.ee/prof_alcunha}}
\item[Research Keywords]
Transport Modelling, Artificial Intelligence, Computer Vision, Urban
Mobility, Accessibility, Vulnerability, Smart Cities, Simulation,
Logistics
\end{description}

\section{\texorpdfstring{EDUCATION
\hrulefill}{EDUCATION }}\label{education}

\begin{enumerate}
\def\labelenumi{\arabic{enumi}.}
\item
  \textbf{Ph.D.~in Transportation Engineering} \hfill Nov.~2013\\
  University of São Paulo (USP), São Carlos School of Engineering
  (EESC), Brazil\\
  \emph{Thesis}:
  ``\texttt{Automatic\ system\ for\ vehicular\ traffic\ parameters\ using\ OpenCV}''\\
  \emph{Advisor}: Prof.~José Reynaldo Anselmo Setti\\
  \emph{DOI}:
  \href{https://doi.org/10.11606/T.18.2013.tde-19112013-165611}{10.11606/T.18.2013.tde-19112013-165611}\\
  Funded by National Council for Scientific and Technological
  Development (CNPq), Brazil.
\item
  \textbf{M.Sc. in Transportation Engineering} \hfill Oct.~2007\\
  University of São Paulo (USP), São Carlos School of Engineering
  (EESC), Brazil\\
  \emph{Thesis}:
  ``\texttt{Evaluation\ of\ performance\ measurement\ impact\ on\ truck\ passenger\ car\ equivalents}''\\
  \emph{Advisor}: Prof.~José Reynaldo Anselmo Setti\\
  \emph{DOI}:
  \href{https://doi.org/10.11606/D.18.2007.tde-27112007-094400}{10.11606/D.18.2007.tde-27112007-094400}\\
  Funded by National Council for Scientific and Technological
  Development (CNPq), Brazil.
\item
  \textbf{B.S. in Civil Engineering} \hfill Feb.~2004\\
  Federal University of Mato Grosso do Sul (UFMS), Campo Grande,
  Brazil\\
  \emph{GPA}: 3.79/4.00 \(\rightarrow\) (9.5/10.0)
\end{enumerate}

\section{\texorpdfstring{EXPERIENCE
\hrulefill}{EXPERIENCE }}\label{experience}

\subsection{Academic Appointments}\label{academic-appointments}

\begin{enumerate}
\def\labelenumi{\arabic{enumi}.}
\item
  \textbf{University of São Paulo (USP-EESC)} \hfill Jul.~2014 --
  present\\
  \emph{Assistant Professor (MS-3.2) \hfill São Carlos, Brazil}\\
  Tenured-track position, Full Dedication to Teaching and Research
  Regime (RDIDP)
\item
  \textbf{University of Zagreb (UNIZG)} \hfill Apr.~2022\\
  \emph{Visiting Lecturer \hfill Zagreb, Croatia}\\
  ERASMUS+ Program: Virtual Teaching Mobility Agreement (Workload: 8h)
\item
  \textbf{University of Melbourne (UniMelb)} \hfill Jan.~2020 -- Dec
  2020\\
  \emph{Visiting Professor \hfill Melbourne, Australia}\\
  CAPES-Print Program -- Junior Visiting Professor
  No.~88887.371506/2019-00
\item
  \textbf{University of Zagreb (UNIZG)} \hfill Jun.~2018\\
  \emph{Visiting Lecturer \hfill Zagreb, Croatia}\\
  ERASMUS+ Program: Higher Education Mobility Agreement (UNIZG/USP-EESC)
  (Workload: 13h)
\item
  \textbf{University of São Paulo (USP)} \hfill Sep.~2017\\
  \emph{Visiting Professor \hfill São Paulo, Brazil}\\
  TUM-USP \texttt{Workshop\ on\ Sustainable\ Mobility} funded by
  BAYLAT/FAPESP Call
\item
  \textbf{University of Minho (UMINHO)} \hfill Jul.~2017\\
  \emph{Visiting Professor \hfill Guimarães, Portugal}\\
  Mission funded by CAPES-FCT n.~39/2014
\item
  \textbf{Technical University of Munich (TUM)} \hfill Nov.~2016 --
  Dec.~2016\\
  \emph{Visiting Professor \hfill Munich, Germany}\\
  TUM-USP \texttt{Workshop\ on\ Sustainable\ Mobility} funded by
  BAYLAT/FAPESP Call
\item
  \textbf{São Paulo State University (UNESP)} \hfill Mar.~2010 --
  Dec.~2010\\
  \emph{Adjunct Professor \hfill Bauru, Brazil}\\
  College of Engineering Bauru (FEB), Civil Engineering undergraduate
  course.
\item
  \textbf{University of São Paulo (USP-EESC)} \hfill Feb.~2009 --
  Jun.~2009\\
  \emph{Graduate Assistant \hfill São Carlos, Brazil}
\item
  \textbf{University of São Paulo (USP-EESC)} \hfill Feb.~2006 --
  Jun.~2006\\
  \emph{Graduate Assistant \hfill São Carlos, Brazil}
\end{enumerate}

\subsection{Professional Experience}\label{professional-experience}

\begin{enumerate}
\def\labelenumi{\arabic{enumi}.}
\item
  \textbf{CCR Highway RioSP (Via Dutra)} \hfill Apr.~2025 -- Nov.~2025\\
  \emph{Technical Consultant -- Transportation Engineering Projects
  \hfill São Paulo, Brazil}\\
  Validate the operational speed of trucks on Via Dutra's new descending
  lane, in Rio de Janeiro (BR-116 highway).
\item
  \textbf{CCR Highway RioSP (Via Dutra)} \hfill Jun.~2023 -- Dec.~2023\\
  \emph{Technical Consultant -- Transportation Engineering Projects
  \hfill São Paulo, Brazil}\\
  Evaluated site conditions to determine optimal placement of truck
  escape ramps on Via Dutra's new descending lane, in Rio de Janeiro
  (BR-116 highway). Simulated operational scenarios to validate design
  effectiveness.
\item
  \textbf{ARTERIS Autopista Litoral Sul (ALS)} \hfill Nov.~2019 --
  Dec.~2019\\
  \emph{Technical Consultant -- Transportation Engineering Projects
  \hfill Curitiba, Brazil}\\
  Directed field testing of BR-376's km 667 truck escape ramp,
  developing protocols and analyzing performance metrics for loaded
  vehicles at multiple approach speeds, with findings implemented in
  concessionaire safety standards\footnote{Interview featured on Rede
    Globo's Jornal Hoje program
    (\url{https://globoplay.globo.com/v/8165879/}).}. Delivered a
  detailed technical assessment of ramp functionality under real-world
  conditions.
\item
  \textbf{University of São Paulo (USP-EESC)} \hfill Feb.~2013 --
  Jun.~2014\\
  \emph{Research Assistant (Laboratory Specialist) \hfill São Carlos,
  Brazil}\\
  Develop scientific research in projects led by faculty, with
  didactic-scientific and extension focus.
\item
  \textbf{Transport Engineering Consultants Ltd.~(TECTRAN)}
  \hfill Apr.~2012 -- Dec.~2012\\
  \emph{Consultant in Transport Planning and Engineering \hfill Belo
  Horizonte, Brazil}\\
  Led the development and integration of structured databases to support
  EPELT, the Transport Logistics Planning Office of the Minas Gerais
  State Secretariat.
\item
  \textbf{Institute of Mathematical and Computer Sciences (ICMC-USP)}
  \hfill Mar.~2012 -- Apr.~2012\\
  \emph{Civil Engineer \hfill São Carlos, Brazil}\\
  Executed AutoCAD-based infrastructure digitization, oversaw routine
  building maintenance, and participated in the supervision of ongoing
  construction projects at ICMC.
\end{enumerate}

\section{\texorpdfstring{TEACHING EXPERIENCE
\hrulefill}{TEACHING EXPERIENCE }}\label{teaching-experience}

\subsection{Lecturer at the University of São Paulo
(USP)}\label{lecturer-at-the-university-of-suxe3o-paulo-usp}

\subsubsection{Undergratuate}\label{undergratuate}

\begin{enumerate}
\def\labelenumi{\arabic{enumi}.}
\item
  \textbf{STT0618 - Air Transport \hfill 2014}\\
  4th year elective transport course in Civil Engineering curriculum.
  Designed the lecturers, exercise and lab sessions. Small classroom of
  10+ students.
\item
  \textbf{STT0403 - Airports, Ports and Waterways
  \hfill 2015--present}\\
  5th year compulsory transport course in Civil Engineering curriculum.
  Designed the lecturers and exercise sessions. Taught in classes of 50+
  students.
\item
  \textbf{STT0408 - Fundamentals of Transportation Engineering
  \hfill 2015--present}\\
  3rd year compulsory transport course in Civil Engineering curriculum.
  Designed and delivered this core transport course, integrating
  lectures, exercises, and applied lab sessions. Taught classes of 50+
  students using inverted classroom strategies and project-based
  learning, fostering active student engagement and applied
  problem-solving. The course received an average student rating of
  4.5/5.0, reflecting strong satisfaction and engagement.
\item
  \textbf{STT0628 - Traffic Engineering and Road Traffic Simulation
  \hfill 2015--present}\\
  3rd year elective transport course in Civil Engineering curriculum.
  Designed the lecturers, exercise and lab sessions. Small classroom of
  10+ students. Presents the fundamental theory of traffic simulation,
  while equipping students to apply concepts in practice and develop key
  technical skills.
\item
  \textbf{1800093 - Final Undergraduate Project \hfill 2016--present}\\
  5th year compulsory transport course in Civil Engineering curriculum.
  My role involves supervising and guiding students through the
  development of their final engineering projects, with a focus on
  applying transport engineering concepts to real-world problems. I
  support students in defining research questions, conducting technical
  analyses, and producing professional-grade reports, while fostering
  independent learning and critical thinking. I have supervised 25+
  projects in this course.
\item
  \textbf{STT0412 - Computational Tools Applied to Civil Engineering
  \hfill 2016--present}\\
  2nd year elective transport course in Civil Engineering curriculum. I
  designed and implemented this course to introduce students to
  computational thinking and practical toolsets for engineering
  problem-solving. The course encourages students to develop programming
  skills and apply digital tools---such as spreadsheets, CAD, GIS, and
  programming languages---to real-world challenges in civil and
  transport engineering. Small classroom of 20+ students.
\item
  \textbf{1800122 - Supervised Internship \hfill 2019--present}\\
  5th year compulsory transport course in Civil Engineering curriculum.
  My role involves supervising and evaluating student internships
  conducted in professional engineering environments. I oversee each
  student's engagement with the host company, assess their performance,
  and ensure that the internship experience aligns with academic and
  professional learning objectives. I have supervised 15+ students.
\item
  \textbf{STT0610 - Logistics and Transportation \hfill 2024--2025}\\
  4th year elective transport course in Civil Engineering curriculum.
  Redesigned course curriculum to address contemporary logistics and
  supply chain challenges: AI-driven logistics tools, GIS-based route
  planning, and Green logistics best practices. Small classroom of 10+
  students.
\item
  \textbf{STT0631 - Logistics in construction \hfill 2026--present}\\
  This elective course integrates theory and practice to prepare
  students for the efficient management of logistical chains in civil
  construction projects. Over the semester, students will develop an
  understanding of the fundamental supply concepts, grasp the scope and
  challenges of providing the necessary resources based on each
  project's scale and characteristics, and learn to identify the factors
  that impact construction logistics --- from cost and scheduling
  concerns to environmental and regulatory constraints.
\item
  \textbf{1800123 - Technical Drawing \hfill 2026--present}\\
  1st year compulsory course in Civil Engineering curriculum. The
  objective of this course is to elucidate the concept and standards of
  design, as well as to present digital tools for Engineering projects
  and the use of georeferenced maps, as well as the use of BIM and 3D
  visualization software. Classroom with 60 students.
\end{enumerate}

\subsubsection{Graduate}\label{graduate}

\begin{enumerate}
\def\labelenumi{\arabic{enumi}.}
\item
  \textbf{STT5874 - Advanced Topics in Traffic Engineering
  \hfill 2015--present}\\
  Elective course in the Transportation Engineering Program. Coordinate
  the course, designed the lectures and lab sessions. Small classroom of
  10+ students. Provides a foundation in traffic simulation theory and
  engages students in applying concepts through real-world scenarios and
  hands-on technical training.
\item
  \textbf{STT5898 - Applied Statistics for Transportation Engineering
  \hfill 2015--present}\\
  Elective course in the Transportation Engineering Program. Coordinate
  the course, designed the lectures and exercises. Small classroom of
  15+ students. This course serves as a foundational milestone,
  equipping students with the core statistical methods required for
  graduate-level study and research.
\item
  \textbf{STT5900 - Multivariate Data Analysis Applied to Transportation
  Engineering \hfill 2015--present}\\
  Elective course in the Transportation Engineering Program. Coordinate
  the course, designed the lectures and exercises. Small classroom of
  15+ students. Course introducing AI techniques using R---such as
  neural networks, clustering, PCA, decision trees, and genetic
  algorithms---applied to each student's own dataset. The course
  culminates in the submission of an article presenting the dataset,
  methodology, and preliminary results.
\item
  \textbf{STT5859 - Transport Technology \hfill 2016--present}\\
  Compulsory course in the Transportation Engineering Program. This core
  course is jointly taught by four professors and provides a
  comprehensive foundation in transportation planning and operations.
  Designed for students at all levels, it offers a structured,
  level-based approach to essential concepts and methodologies in the
  field. Small classroom of 15+ students.
\item
  \textbf{STT5905 - Bibliographic Research for Transportation Systems
  \hfill 2017--present}\\
  Compulsory course in the Transportation Engineering Program. A core
  course that guides and encourages students to develop a comprehensive
  literature review, fostering critical analysis and familiarity with
  key academic sources in the field. Small classroom of 15+ students.
\item
  \textbf{STT5909 - Data Analysis Laboratory with Open-Source Software R
  \hfill 2017}\\
  Elective course in the Transportation Engineering Program. Coordinate
  the course, designed the lectures and exercises. Small classroom of
  10+ students. This course was designed to provide a foundational
  introduction to R programming for solving transport engineering
  problems.
\end{enumerate}

\section{\texorpdfstring{SUPERVISION
\hrulefill}{SUPERVISION }}\label{supervision}

\begin{itemize}
\tightlist
\item
  \textbf{PhD Students}: 6 (2 completed, 4 ongoing)
\item
  \textbf{MSc Students}: 17 (13 completed, 4 ongoing)
\item
  \textbf{Scientific Initiation}: 16 (14 completed, 2 ongoing)
\item
  \textbf{Undergraduate Projects}: 25 completed
\end{itemize}

\section{\texorpdfstring{RESEARCH INCOME
\hrulefill}{RESEARCH INCOME }}\label{research-income}

I have secured over \textbf{BRL 18,600,000}\footnote{Exchange rates
  used: EUR 1.00 \(\approx\) BRL 6.30; USD 1.00 \(\approx\) BRL 5.50;
  AUD 1.00 \(\approx\) BRL 3.50.} (approximately USD 3,381,818
\textbar{} EUR 2,952,381 \textbar{} AUD 5,314,286) in research funding,
during my tenure at USP, with projects spanning intelligent transport
systems and sustainable mobility solutions.

\begin{enumerate}
\def\labelenumi{\arabic{enumi}.}
\item
  \textbf{CCD Sustainability and Innovation in Road Infrastructure}
  \hfill BRL 8,000,000\\
  \emph{Pavement Recycling as a Pillar of Decarbonization - Centers for
  Science for Development (CCD)} \hfill 2025--2030\\
  Role: Co-Principal Investigator\\
  Sponsor: FAPESP Grant (2025/07146-8), Brazil
\item
  \textbf{Redesign and Validate the Truck Escape Ramp on the BR-116 (Via
  Dutra)} \hfill BRL 180,000\\
  Role: Principal Investigator \hfill 2025--2025\\
  Sponsor: Group CCR Highways (RioSP), Brazil
\item
  \textbf{Artificial Intelligence Recreating Environments (IARA)}
  \hfill BRL 10,000,000\\
  \emph{Applied Research Centers Program (CEPID)} \hfill 2023--2028\\
  Role: Research Collaborator\\
  Sponsor: FAPESP Grant (Process 20/09835-1), Brazil
\item
  \textbf{Rethinking traffic modeling in transport networks for the next
  generation of smart/connected cities} \hfill BRL 72,000\\
  Role: Principal Investigator \hfill 2023--2026\\
  Sponsor: CNPq Consolidated Research Groups Grant (Process
  409087/2023-8), Brazil
\item
  \textbf{Artificial Intelligence: development of tools for urban
  mobility} \hfill BRL 40,000\\
  Role: Principal Investigator \hfill 2023--2026\\
  Sponsor: CNPq Research Productivity Grant (Process 311964/2022-2),
  Brazil
\item
  \textbf{Evaluation of Truck Escape Ramp on the BR-116 (Via Dutra)}
  \hfill BRL 90,000\\
  Role: Principal Investigator \hfill 2023--2023\\
  Sponsor: Group CCR Highways (RioSP), Brazil
\item
  \textbf{Innovative Control Strategies for Sustainable Mobility in
  Smart Cities} \hfill EUR 8,000\\
  Role: Co-Principal Investigator \hfill 2021--2021\\
  Sponsor: University of Zagreb (UNIZG), Croatia
\item
  \textbf{Visiting Professorship} \hfill AUD 150,000\\
  Role: Principal Investigator \hfill 2019--2019\\
  Sponsor: CAPES-Print Program (Process 88887.371506/2019-00), Brazil
\item
  \textbf{Site Optimization for Truck Escape Ramps on the BR-376}
  \hfill BRL 150,000\\
  Role: Principal Investigator \hfill 2019--2019\\
  Sponsor: ARTERIS Autopista Litoral Sul (ALS), Brazil
\item
  \textbf{Image-based method for axle detection and truck
  classification} \hfill BRL 15,000\\
  Role: Principal Investigator \hfill 2018--2022\\
  Sponsor: CNPq Universal Grant (Process 436954/2018-4), Brazil
\item
  \textbf{Application of deep learning in intelligent traffic control
  system} \hfill EUR 8,500\\
  Role: Co-Principal Investigator \hfill 2018--2018\\
  Sponsor: University of Zagreb (UNIZG), Croatia
\item
  \textbf{Studies aimed at promoting sustainable and safe urban
  mobility} \hfill BRL 20,000\\
  Role: Co-Principal Investigator \hfill 2013--2016\\
  Sponsor: CAPES/FCT Program (Process 39/2017), Brazil
\end{enumerate}

\section{\texorpdfstring{PUBLICATIONS
\hrulefill}{PUBLICATIONS }}\label{publications}

\subsection{Submitted Manuscripts}\label{submitted-manuscripts}

\begin{enumerate}
\def\labelenumi{\arabic{enumi}.}
\item
  LOURO, T.V.; GRIGILON, A.B.; TIRACHINI, A.; \textbf{CUNHA, A.L.};
  GEURS, K.T. (2025) ``How Do E-Bikes Measure Up? Analyzing Speed
  Differences and Network Impacts of São Paulo's Bikesharing System''.
  Transportation. \textless DOI:
  \href{https://dx.doi.org/10.2139/ssrn.5648290}{10.2139/ssrn.5648290}\textgreater{}
\item
  SALINAS, K., BARELLA, V., \textbf{CUNHA, A.L.}, OLIVEIRA, G.M., VIERA,
  T., NONATO, L.G. (2025) ``ORDENA: ORigin-DEstiNAtion data
  exploration''. IEEE Transactions on Visualization and Computer
  Graphics
  \textless{}\href{https://arxiv.org/pdf/2510.18278}{arXiv:2510.18278}\textgreater{}
\end{enumerate}

\subsection{Peer-Reviewed Journal}\label{peer-reviewed-journal}

\begin{enumerate}
\def\labelenumi{\arabic{enumi}.}
\item
  LOURO, T.V.; GRIGILON, A.B.; \textbf{CUNHA, A.L.}; GEURS, K.T. (2025)
  ``E-bikes' impact on job accessibility and equity in São Paulo and
  Rio''. Transportation Research Part D: Transport and Environment.
  \textless DOI:
  \href{https://doi.org/10.1016/j.trd.2025.105072}{10.1016/j.trd.2025.105072}\textgreater{}
\item
  DE OLIVEIRA, G.J.M.; LAVIERI, P.S.; \textbf{CUNHA, A.L.} (2023)
  Integrating a non-gridded space representation into a graph neural
  networks model for citywide short-term crash risk prediction. Urban
  Informatics. v.2, p.7.\\
  \textless DOI:
  \href{https://doi.org/10.1007/s44212-023-00032-6}{10.1007/s44212-023-00032-6}\textgreater{}
\item
  FLEURY, M.P.; KAMAKURA, G.K.; PITOMBO, C.S.; \textbf{CUNHA, A.L.B.N.};
  FERREIRA, F.B.; LINS DA SILVA, J. (2023) Assessing and Predicting
  Geogrid Reduction Factors after Damage Induced by Dropping Recycled
  Aggregates. Sustainability. v.15, p.9942.\\
  \textless DOI:
  \href{https://doi.org/10.3390/su15139942}{10.3390/su15139942}\textgreater{}
\item
  FLEURY, M.P.; KAMAKURA, G.K.; PITOMBO, C.S.; \textbf{CUNHA, A.L.B.N.};
  LINS DA SILVA, J. (2023) Prediction of non-woven geotextiles'
  reduction factors for damage caused by the drop of backfill materials.
  Geotextiles and Geomembranes. v.1, p.1 - 11.\\
  \textless DOI:
  \href{https://doi.org/10.1016/j.geotexmem.2023.05.004}{10.1016/j.geotexmem.2023.05.004}\textgreater{}
\item
  SILVA, F.A.E.; BESSA JUNIOR, J.E.; COSTA, A.L.; \textbf{CUNHA, A.L.};
  VELHO, D.M.C.; ANDALICIO, A. (2023) Exploratory analysis of the VISSIM
  simulation model for two-lane highways. Engenharia Civil UM (Braga),
  n.63, p.6-17.\\
  \textless DOI:
  \href{https://doi.org/10.21814/ecum.4493}{10.21814/ecum.4493}\textgreater{}
\item
  SILVA, F.A.; BESSA JUNIOR, J.E.; COSTA, A.L.; \textbf{CUNHA, A.L.};
  VELHO, D.M.C. (2022) Analysis of no-passing zones to assess the level
  of service on two-lane rural highways in Brazil. Case Studies on
  Transport Policy. v.10, p.248-256.\\
  \textless DOI:
  \href{https://doi.org/10.1016/j.cstp.2021.12.006}{10.1016/j.cstp.2021.12.006}\textgreater{}
\item
  MORELLI, A. B.; \textbf{CUNHA, A.L.} (2021) Assessing vulnerabilities
  in transport networks: a graph-theoretic approach. Transportes (Rio de
  Janeiro). v.29, p.161-172.\\
  \textless DOI:
  \href{https://doi.org/10.14295/transportes.v29i1.2250}{10.14295/transportes.v29i1.2250}\textgreater{}
\item
  SILVA, F.A.; BESSA JÚNIOR, J.E.; COSTA, A.L.; \textbf{CUNHA, A.L.};
  ANDALÍCIO, A.F.; DA COSTA VELHO, D.M.; NAZARETH, V.S. (2021)
  Evaluation of the effect of climbing lanes on segments of two-lane
  highways. Transportes (Rio de Janeiro). v.29, p.1-16.\\
  \textless DOI:
  \href{https://doi.org/10.14295/transportes.v29i1.2359}{10.14295/transportes.v29i1.2359}\textgreater{}
\item
  MORELLI, A.B.; \textbf{CUNHA, A.L.} (2021) Measuring urban road
  network vulnerability to extreme events: An application for urban
  floods. Transportation Research Part D -- Transport and Environment.
  v.93, p.102770.\\
  \textless DOI:
  \href{https://doi.org/10.1016/j.trd.2021.102770}{10.1016/j.trd.2021.102770}\textgreater{}
\item
  MARTINS, D.O.; OLIVEIRA, G.J.M.; MORAES, F.R.; SILVA, I.;
  \textbf{CUNHA, A.L.} (2020) Geomatics data management system. Revista
  Brasileira de Geomática. v.8, p.056-069.\\
  \textless DOI:
  \href{https://doi.org/10.3895/rbgeo.v8n1.10141}{10.3895/rbgeo.v8n1.10141}\textgreater{}
\item
  PIANUCCI, M.N.; PITOMBO, C.S.; \textbf{CUNHA, A.L.}; LIMA SEGANTINE,
  P.C. (2019) Forecasting household travel demand through a sequential
  method based on synthetic population and artificial neural networks.
  Transportes (Rio de Janeiro). v.27, p.1-23.\\
  \textless DOI:
  \href{https://doi.org/10.14295/transportes.v27i4.1409}{10.14295/transportes.v27i4.1409}\textgreater{}
\item
  OLIVEIRA, J.V.M.; LAROCCA, A.P.C.; ARAUJO NETO, J.O.; \textbf{CUNHA,
  A.L.}; SANTOS, M.C.; SCHAAL, R.E. (2019) Rigid Bridges Health Dynamic
  Monitoring Using 100 Hz GPS Single-Frequency and Accelerometers.
  Positioning. v.10, p.17-33.\\
  \textless DOI:
  \href{https://doi.org/10.4236/pos.2019.102002}{10.4236/pos.2019.102002}\textgreater{}
\item
  DE OLIVEIRA, J.V.M.; LAROCCA, A.P.C.; DE ARAÚJO NETO, J.O.; CUNHA,
  A.L.; DOS SANTOS, M.C.; SCHAAL, R.E. (2019) Vibration monitoring of a
  small concrete bridge using wavelet transforms on GPS data. Journal Of
  Civil Structural Health Monitoring. v.9, p.397-409.\\
  \textless DOI:
  \href{https://doi.org/10.1007/s13349-019-00341-y}{10.1007/s13349-019-00341-y}\textgreater{}
\item
  LINDNER, A.; PITOMBO, C.S.; \textbf{CUNHA, A.L.} (2017) Estimating
  motorized travel mode choice using classifiers: An application for
  high-dimensional multicollinear data. Travel Behaviour and Society.
  v.6, p.100-109.\\
  \textless DOI:
  \href{https://doi.org/10.1016/j.tbs.2016.08.003}{10.1016/j.tbs.2016.08.003}\textgreater{}
\item
  SOUZA, N.C.; PITOMBO, C.; \textbf{CUNHA, A.L.}; LAROCCA, A.P.C.; DE
  ALMEIDA FILHO, G.S. (2017) Model for classification of linear erosion
  processes along railways through decision tree algorithm and
  geotechnologies. Boletim de Ciências Geodésicas. v.23, p.72-86.\\
  \textless DOI:
  \href{https://doi.org/10.1590/S1982-21702017000100005}{10.1590/S1982-21702017000100005}\textgreater{}
\item
  ANDRADE, G.R.; PITOMBO, C.; \textbf{CUNHA, A.L.N.}; SETTI, J.R. (2016)
  A Model for Estimating Free-Flow Speed on Brazilian Expressways.
  Transportation Research Procedia. v.15, p.378-388.\\
  \textless DOI:
  \href{https://doi.org/10.1016/j.trpro.2016.06.032}{10.1016/j.trpro.2016.06.032}\textgreater{}
\item
  LAROCCA, A.P.C.; ARAÚJO NETO, J.O.; TRABANCO, J.L.A.; BARBOSA, A.C.B.;
  \textbf{CUNHA, A.L.B.N.}; SCHAAL, R.E. (2015) Use of 100 Hz GPS
  receivers in the detection of millimeter vertical deflections of small
  concrete bridges. Boletim de Ciências Geodésicas. v.21, p.290-307.\\
  \textless DOI:
  \href{https://doi.org/10.1590/S1982-21702015000200017}{10.1590/S1982-21702015000200017}\textgreater{}
\item
  LAROCCA, A.P.C.; ARAUJO NETO, J.O.; BARBOSA, A.C.B.; TRABANCO, J.L.A.;
  \textbf{CUNHA, A.L.B.N.} (2014) Dynamic Monitoring vertical Deflection
  of Small Concrete Bridge Using Conventional Sensors And 100 Hz GPS
  Receivers - Preliminary Results. IOSRJEN Journal of Engineering. v.04,
  p.09-20.\\
  \textless DOI:
  \href{https://doi.org/10.9790/3021-04920920}{10.9790/3021-04920920}\textgreater{}
\item
  \textbf{CUNHA, A.L.}; SETTI, J.R. (2011) Truck equivalence factors for
  divided, multilane highways in Brazil. Procedia: Social and Behavioral
  Sciences. v.16, p.248-258.\\
  \textless DOI:
  \href{https://doi.org/10.1016/j.sbspro.2011.04.447}{10.1016/j.sbspro.2011.04.447}\textgreater{}
\end{enumerate}

\subsection{Conference Proceedings}\label{conference-proceedings}

\begin{enumerate}
\def\labelenumi{\arabic{enumi}.}
\item
  MORELLI, A.B.; ALIZON, G.B.; \textbf{CUNHA, A.L.} (2025)
  Alternative-Route Efficiency in Brazilian Cities: How Flood-Induced
  Collapse Patterns Differ from Random Blockages. In: XXXIX ANPET --
  Research and Teaching in Transport Congress, 2025, Goiânia.
  Proceedings of the 39th ANPET.
\item
  LOURO, T.V.; ASSIS, L.B.M.; JUNIOR, J.U.P.; \textbf{CUNHA, A.L.};
  GEURS, K.T (2025) Job Accessibility in the 15-Minute City: A
  Comparative Analysis of Walking, Cycling, and E-Bikes in Four
  Brazilian Cities. In: XXXIX ANPET -- Research and Teaching in
  Transport Congress, 2025, Goiânia. Proceedings of the 39th ANPET.
\item
  ISHIHARA, B.A.; QUINTINO, P.G.; \textbf{CUNHA, A.L.}; SETTI, J.R.
  (2025) Operation of Heavy Vehicles on Long, Steep Downgrades: Brake
  Thermal Simulation Based on ABNT NBR 10966-2. In: XXXIX ANPET --
  Research and Teaching in Transport Congress, 2025, Goiânia.
  Proceedings of the 39th ANPET.
\item
  RODRIGUES, L.R.; PITOMBO, C.S.; \textbf{CUNHA, A.L.}; LAROCCA, A.P.C.;
  FERRAZ, A.C.P. (2025) Identification of Crime and Traffic Crash
  Hotspots in the Vicinity of Bus Stops. In: XXXIX ANPET -- Research and
  Teaching in Transport Congress, 2025, Goiânia. Proceedings of the 39th
  ANPET.
\item
  DAVOLI, J.P.; PITOMBO, C.S.; \textbf{CUNHA, A.L.} (2025) Application
  of a Random-Forest--Based Variable Selection Method for the Analysis
  of Post-COVID-19 Active Transportation. In: XXXIX ANPET -- Research
  and Teaching in Transport Congress, 2025, Goiânia. Proceedings of the
  39th ANPET.
\item
  MORELLI, A.B.; \textbf{CUNHA, A.L.} (2024) Vulnerability to flooding:
  how long-trip prevalence reduces the efficiency of alternative routes.
  In: XXXVIII ANPET -- Research and Teaching in Transport Congress,
  2024, Florianópolis. Proceedings of the 38th ANPET.
\item
  MARCOMINI, L.A.; \textbf{CUNHA, A.L.} (2023) Truck axle detection
  using Neural Networks: analysis of the number of images in the
  training dataset. In: ANPET -- Research and Teaching in Transport
  Congress, 2023, Santos. Proceedings of the 37th ANPET.
\item
  MORELLI, A.B.; LOURO, T.V.; \textbf{CUNHA, A.L.} (2022) Proposal of
  bikeability indicators from an accessibility perspective: identifying
  roads best suited for cycle lanes using widely available data. In:
  XXXVI ANPET -- Research and Teaching in Transport Congress, 2022,
  Fortaleza. Proceedings of the XXXVI ANPET.
\item
  MORELLI, A.B.; \textbf{CUNHA, A.L.} (2021) Pedestrian accessibility:
  impacts of morphological and demographic characteristics on access to
  facilities. In: XXXV ANPET -- Research and Teaching in Transport
  Congress, 2021. Proceedings of the XXXV ANPET.
\item
  OLIVATTO, T.F.; PITOMBO, C.S.; \textbf{CUNHA, A.L.}; MELANDA, E.A.
  (2020) Relationships between the nutritional status of preschoolers
  and socioeconomic and urban infrastructure indicators: a CART-based
  approach. In: PLURIS -- Luso-Brazilian Congress on Urban, Regional,
  Integrated and Sustainable Planning, 2020. Proceedings of the 9th
  PLURIS.
\item
  BOSCO JUNIOR, A.D.; \textbf{CUNHA, A.L.} (2020) Street and zonal scale
  relationship between network centrality and economic activities: case
  study in Curitiba, Brazil. In: PLURIS -- Luso-Brazilian Congress on
  Urban, Regional, Integrated and Sustainable Planning, 2020.
  Proceedings of the 9th PLURIS.
\item
  VIZIOLI, H.T.; KUŠIC, K.; IVANJKO, E.; \textbf{CUNHA, A.L.} (2020) A
  method to calibrate Variable Speed Limit Control on high-truck-share
  roads: a case study in Brazil. In: Brazilian Technology Symposium -
  BTSym'20, 2020, Campinas. Smart Innovation, Systems and Technologies.
\item
  CAKIJA, D.; ASSIRATI, L.; IVANJKO, E.; \textbf{CUNHA, A.L.} (2019)
  Autonomous Intersection Management: A Short Review. In: 61st
  International Symposium ELMAR-2019, Zadar. Proceedings of the 61st
  ELMAR Symposium.
\item
  MORELLI, A.B.; \textbf{CUNHA, A.L.} (2019) Identifying vulnerabilities
  in transport networks: a graph-theoretical approach. In: XXXIII ANPET
  -- Research and Teaching in Transport Congress, 2019, Balneário
  Camboriú. Proceedings of the XXXIII ANPET.
\item
  KURAMOTO, B.; \textbf{CUNHA, A.L.} (2019) Usability and limitations of
  collaborative map data in accessibility analysis. In: XXXIII ANPET --
  Research and Teaching in Transport Congress, 2019, Balneário Camboriú.
  Proceedings of the XXXIII ANPET.
\item
  SILVA, F.A.E.; BESSA JUNIOR, J.E.; COSTA, A.L.; \textbf{CUNHA, A.L.};
  ANDALICIO, A.F.; VELHO, D.M.C.; NAZARETH, V.S. (2019) Evaluation of
  the effect of climbing lanes on single-lane highway segments. In:
  XXXIII ANPET -- Research and Teaching in Transport Congress, 2019,
  Balneário Camboriú. Proceedings of the XXXIII ANPET.
\item
  MARCOMINI, L.A.; \textbf{CUNHA, A.L.} (2019) The impact of different
  video resolutions in a feature-based vehicle detection algorithm. In:
  XXXIII ANPET, 2019, Balneário Camboriú. Proceedings of XXXIII ANPET.
\item
  MORELLI, A.B.; \textbf{CUNHA, A.L.} (2019) A strategy for assessing
  the impact of flooding on urban road systems. In: XXXIII ANPET --
  Research and Teaching in Transport Congress, 2019, Balneário Camboriú.
  Proceedings of the XXXIII ANPET.
\item
  KURAMOTO, B.; \textbf{CUNHA, A.L.} (2018) Methodological proposal for
  construction and analysis of a real urban network. In: PLURIS -- 8th
  Luso-Brazilian Congress on Urban, Regional, Integrated and Sustainable
  Planning, 2018, Coimbra. Proceedings of the 8th PLURIS.
\item
  MORELLI, A.B.; \textbf{CUNHA, A.L.} (2018) Methods for evaluating
  traffic conditions using Google Traffic and Twitter data. In: XXXII
  ANPET, 2018, Gramado, RS. Proceedings of XXXII ANPET.
\item
  THEBIT, M.M.; \textbf{CUNHA, A.L.} (2017) Comparison of traffic data
  available on the web and obtained by fixed sensors. In: XXXI ANPET,
  2017, Recife. Proceedings of XXXI ANPET.
\item
  RIBEIRO, E.R.; \textbf{CUNHA, A.L.} (2017) Exploratory analysis of a
  method for anomaly detection in traffic data using Wavelet. In: XXXI
  ANPET -- Research and Teaching in Transport Congress, 2017, Recife.
  Proceedings of the XXXI ANPET.
\item
  PANICE, N.R.; \textbf{CUNHA, A.L.} (2017) Evaluation of a method for
  automatic truck axle detection in images. In: XXXI ANPET -- Research
  and Teaching in Transport Congress, 2017, Recife. Proceedings of the
  XXXI ANPET.
\item
  OLIVEIRA, G.J.M.; \textbf{CUNHA, A.L.} (2017) HCM calibration method
  for dual carriageways and expressways using Bayesian inference. In:
  XXXI ANPET -- Research and Teaching in Transport Congress, 2017,
  Recife. Proceedings of the XXXI ANPET.
\item
  ASSIS, R.K.M.; DURAN, J.B.C.; \textbf{CUNHA, A.L.}; PITOMBO, C.S.;
  FERNANDES JUNIOR, J.L. (2016) Application of Artificial Neural
  Networks for predictive modeling of pavement functional
  classification. In: PLURIS -- Luso-Brazilian Congress on Urban,
  Regional, Integrated and Sustainable Planning, 2016, Maceió.
  Proceedings of the 7th PLURIS.
\item
  FERREIRA, F.A.; PITOMBO, C.S.; \textbf{CUNHA, A.L.} (2016) Forecasting
  mode choice on a university campus using binomial logistic regression.
  In: PLURIS -- Luso-Brazilian Congress on Urban, Regional, Integrated
  and Sustainable Planning, 2016, Maceió. Proceedings of the 7th PLURIS.
\item
  THEBIT, M.M.; \textbf{CUNHA, A.L.}; PITOMBO, C.S. (2016) Relationship
  between bus mode supply and modal choice for airport access: a data
  mining approach. In: PLURIS -- Luso-Brazilian Congress on Urban,
  Regional, Integrated and Sustainable Planning, 2016, Maceió.
  Proceedings of the 7th PLURIS.
\item
  RIBEIRO, E.R.; \textbf{CUNHA, A.L.} (2016) Exploratory analysis of a
  method for defining a typical day using Wavelet transform and cluster
  analysis. In: XXX ANPET -- Research and Teaching in Transport
  Congress, 2016, Rio de Janeiro, RJ. Proceedings of the XXX ANPET.
\item
  ANDRADE, G.R.; PITOMBO, C.S.; \textbf{CUNHA, A.L.B.N.}; SETTI, J.R.;
  FERRAZ, A.C.P. (2015) Forecasting free-flow speed on São Paulo
  expressways and highways. In: 9th Brazilian Congress of Highways and
  Concessions -- CBR\&C, 2015, Brasília. Proceedings of the 9th CBR\&C.
\item
  ROCHA, S.S.; PIANUCCI, M.N.; PITOMBO, C.S.; \textbf{CUNHA, A.L.B.N.}
  (2015) Use of Neural Networks for trip production forecasting: an
  aggregate analysis. In: XXIX ANPET -- Research and Teaching in
  Transport Congress, 2015, Ouro Preto, MG. Proceedings of the XXIX
  ANPET.
\item
  \textbf{CUNHA, A.L.B.N.}; SETTI, J.R.; GONZAGA, A. (2013) Comparison
  of background generation models in vehicular traffic video processing.
  In: XXVII ANPET -- Research and Teaching in Transport Congress, 2013,
  Belém. Proceedings of the XXVII ANPET.
\item
  BESSA JUNIOR, J.E.; \textbf{CUNHA, A.L.B.N.}; SETTI, J.R. (2011)
  Comparison between CORSIM and TWOPAS simulators for modeling two-lane
  highways. In: XXV ANPET -- Research and Teaching in Transport
  Congress, 2011, Rio de Janeiro, RJ. National Overview of Transport
  Research 2011, p.2140--2151.
\item
  \textbf{CUNHA, A.L.B.N.}; SETTI, J.R. (2009) Equivalence factors for
  trucks on dual carriageway highways. In: 6th Brazilian Congress of
  Highways and Concessions -- CBR\&C, 2009, Florianópolis, SC.
  Proceedings of the 6th CBR\&C.
\item
  \textbf{CUNHA, A.L.B.N.}; MODOTTI, M.M.; SETTI, J.R. (2008) Truck
  classification through cluster analysis. In: XXII ANPET -- Research
  and Teaching in Transport Congress, 2008, Fortaleza, CE. Proceedings
  of the XXII ANPET.
\item
  BESSA JUNIOR, J.E.; LIMA, F.A.A.; \textbf{CUNHA, A.L.B.N.}; SETTI,
  J.R. (2008) Calibration of the Integration simulator's performance
  model using a genetic algorithm. In: XXII ANPET -- Research and
  Teaching in Transport Congress, 2008, Fortaleza, CE. Proceedings of
  the XXII ANPET.
\item
  \textbf{CUNHA, A.L.B.N.}; SETTI, J.R. (2006) Calibration of the CORSIM
  truck performance model using a genetic algorithm. In: XX ANPET --
  Research and Teaching in Transport Congress, 2006, Brasília.
  Proceedings of the XX ANPET, v.I, p.88--99.
\end{enumerate}

\section{\texorpdfstring{AWARDS \& HONORS
\hrulefill}{AWARDS \& HONORS }}\label{awards-honors}

\begin{itemize}
\item
  \textbf{ANPET Scientific Production Award} \hfill 2023\\
  National Agency for Transportation Research and Education (ANPET),
  Brazil.
\item
  \textbf{Excellence Certificate} \hfill 2017\\
  Best professor of the Department of Transportation Engineering
  (USP-EESC-STT),\\
  Academic Secretariat of Civil Engineering (SACivil), Brazil.
\item
  \textbf{Excellence Certificate} \hfill 2016\\
  Best professor of the Department of Transportation Engineering
  (USP-EESC-STT),\\
  Academic Secretariat of Civil Engineering (SACivil), Brazil.
\item
  \textbf{ABCR Innovation Salon Award} \hfill 2015\\
  9th Brazilian Congress on Highways and Concessions (CBR\&C),\\
  5th Innovation Salon of the Brazilian Association of Highway
  Concessionaires (ABCR), Brazil.
\end{itemize}

\section{\texorpdfstring{PROFESSIONAL SERVICES
\hrulefill}{PROFESSIONAL SERVICES }}\label{professional-services}

\begin{itemize}
\item
  \textbf{Academic Service}: Member of Department Council of Transport
  Engineering, Research and Innovation Committee (CPqI), Graduate
  Program Coordination Committee in Transport Engineering (CCP-ET),
  Culture and University Extension Committee (CCEx), Center for
  Educational Technology in Engineering (CETEPE),
\item
  \textbf{Reviewer}: Transportation Research Part E, Sustainability,
  Case Studies on Transport Policy, Transportes, Sensors, Promet -
  Traffic \& Transportation Journal, Geo-spatial Information Science,
  Journal of the International Association of Traffic and Safety
  Sciences (IATSS), Drones, Engineering Applications of Artificial
  Intelligence (EAAI), Sensors, Sustainable Cities and Society.
\item
  \textbf{Scientific Committee}: Transportation Research Board TRB, IEEE
  Intelligent Transportation Systems Society (ITSS), National
  Association for Research and Education in Transportation (ANPET),
  International Scientific Conference (ZIRP), International Symposium
  ELMAR, The Science and Development of Transport (TRANSCODE).
\end{itemize}

\section{\texorpdfstring{TECHNICAL SKILLS
\hrulefill}{TECHNICAL SKILLS }}\label{technical-skills}

\begin{itemize}
\item
  \textbf{Programming languages}: C++, R, Python, Julia, HTML, CSS,
  JavaScript, Matlab
\item
  \textbf{Tools}: CAD, Civil-3D, OpenCV, TSIS-CORSIM, AIMSUN, SUMO,
  VISSIM, MATSim, QGIS
\item
  \textbf{Languages}: Portuguese (native), English (advanced), Spanish
  (basic)
\end{itemize}




\end{document}
